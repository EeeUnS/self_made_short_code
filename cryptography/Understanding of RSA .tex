\documentclass{oblivoir}
    \usepackage{ikps,ansform}
    \usepackage{lipsum}
  
    \newcounter{problem}[section]
    \newenvironment{problem}{\noindent\refstepcounter{problem}\textbf{\large\theproblem.} }{}
    
\begin{document}
\par
\title{RSA시스템의 기반 수학과 이해}
\author{ EUnS }
\maketitle

\chapter{}
\begin{justbox}
$d,m,n,$이 어떤 정수일 때 \par
1. $d$가 $m$과 $n$의 공약수일때, $m+n$도 $d$의 배수이다.
\par
2. $d$가 $m$과 $n$의 공약수일때, $m-n$도 $d$의 배수이다.
\end{justbox}
이에 대한 증명은 간단합니다. \par
$m=dq_1 , n = dq_2$( $q_1,q_2$는 어떤 정수)\par
$m+n = d(q_1 + q_2) , m-n=d(q_1-q_2)$ \par
참고로 $d$가 $n$의 약수(인수)일 때 $d\: |\: n$으로 표시합니다.
\par 
$m$과 $n$의 최대공약수는 $\gcd(m,n)$이라고 합니다.\par
$r$이 $a$를 $b$로 나눈 나머지라면  $r=a\bmod b$입니다. \par
이를써서 위 명제를 다시 적으면 $d\mid n , d\mid m  \longrightarrow d \mid (m+n), d\mid (m-n)$
\newpage
\begin{justbox}
$a,b,z$를 양의 정수라 하면, 다음이 성립한다.
\begin{center}
    $ab\bmod z= [(a\bmod z)(b \bmod z)]\bmod z$
\end{center}
\end{justbox}
$w = ab\bmod z, x =a \bmod z, y=b\bmod z$라 하자.
다음이 성립하는 $q_1$이 존재한다.
\begin{center}
    $ab=q_1z+w \Longleftrightarrow w=ab-q_1 z$
\end{center}
마찬가지로 다음을 만족시키는 $q_2$와 $q_3$가 존재한다.
\begin{center}
    $a=q_2 z + x , b=q_3 +y$
    \par
    $w = ab-q_1 z = (q_2z+x)(q_3z+y)-q_1z$\par $=(q_2q_3z+q_2y+q_3x-q_1)z+xy$\par
    $=qz+xy$
\end{center}
여기서 $q=q_2q_3z+q_2y+q_3x-q_1$이므로 
\begin{center}
    $xy=-qz+w$
\end{center}
즉 $w$는 $xy$를 $z$로 나눌 때의 나머지이다. 그러므로 $w=xy \bmod z$가 되고 이는 다음과 같이 변환된다.
\begin{center}
     $ab\bmod z= [(a\bmod z)(b \bmod z)]\bmod z$
\end{center}
이는 큰수를 인수분해해서 작은값으로 나눠서 큰수를 다루는 부담을 덜어주지만 지수승에 대해서도 응용이 가능하다.
이를 이용해서 $a^{29}\bmod z$를 계산하는 절차를 예시로 들어보겠다. $a^{29}$는 다음과 같은 순서로 계산한다.
\begin{center}
    $a$ , $a^{5}=a \cdot a^4$, $a^{13}=a^{5}\cdot a^{8}$, $a^{29}=a^{13}\cdot a^{16}$
\end{center}
$a^{29} \bmod z$는 다음과 같은 순서로 계산한다.
\begin{center}
    $a \bmod z$ , $a^{5}\bmod z$, $a^{13}\bmod z$, $a^{29}\bmod z$
\end{center}
\begin{center}
    $a^2 \bmod z=[(a\bmod z)(a\bmod z)]\bmod z$ \par
    $a^4 \bmod z=[(a^2\bmod z)(a^2\bmod z)]\bmod z$ \par
    $a8 \bmod z=[(a^4\bmod z)(a^4\bmod z)]\bmod z$ \par
    $a^{16} \bmod z=[(a^8\bmod z)(a^8\bmod z)]\bmod z$ \par
    $a^5 \bmod z=[(a\bmod z)(a^4\bmod z)]\bmod z$ \par
    $a^{13} \bmod z=[(a^5\bmod z)(a^8\bmod z)]\bmod z$ \par
    $a^{29} \bmod z=[(a^{13}\bmod z)(a^{16}\bmod z)]\bmod z$ \par
\end{center}

\newpage
\chapter{}
\section{유클리드 호제법(Euclidean algorithm)}
\begin{justbox}
$a$가 음이 아닌 정수이고, $b$가 양의 정수이며, $r$이 $a$를 $b$로 나눈 나머지라면 $a$와 $b$의 최대공약수는 $b$와 $r$의 최대공약수와 같다.
\end{justbox}
위 명제를 위에 적었던 표기법을 사용하면 \par
$a$가 음이 아닌 정수이고, $b$가 양의 정수이며 $r=a\:\bmod \: b$이면 $\gcd(a,b) = \gcd(b,r)$이다.
\par
뭐 어쨋든, 증명을 하자면 $a=bq +r (0 \le r\: <\: b , q$ 는 어떤 정수 )인데, $c$를 $a$와  $b$의 공약수라 하면, $c$는 $bq$의 약수인 것은 자명합니다. $a$또환 $c$의 약수이므로 $c$는 $a-bq\:(=r)$의 약수입니다. 따라서 $c$는 $b$와 $r$의 공약수입니다. 반대로 $c'$가 $b$와 $r$의 공약수이면, $c'$는 $bq+r(=a)$의 약수가 되고 따라서 $a$와 $b$의 공약수가 됩니다. 따라서 $a$와 $b$의 공약수 집합이 $b$와 $r$의 공약수 집합과 같으므로 $\gcd(a,b) = \gcd(b,r)$이 성립합니다.\par

\vspace{1\baselineskip}

유클리드 알고리즘의 의의는 나머지 연산만을 이용해서 뺑뺑이 돌리면 어떻게 됬든지 간에 최대공약수를 기계적으로 구할수있다는 것에 있다. $\gcd(a,b) = \gcd(b,r)$에서 b,r을 새로운 a,b로서 값을 넣어서 연속적으로 계산을 하면 언젠가 b가 0이 되는 순간이 오는데 이때 a가 처음 a,b의 최대 공약수가 되는것이다.
\newpage

\section{확장된 유클리드 알고리즘(Extended Euclidean algorithm)} 
확장된 유클리드 알고리즘은 다음의 명제에대해서 $s$와 $t$를 효율적으로 구하는 방법에대한 것이다. 
\begin{justbox}
$a$와 $b$가 음이 아니고 동시에 0이 아닌 정수라 하면 다음을 만족시키는 정수 $s$와 $t$가 존재한다.
\begin{center}
    $\gcd(a,b) = s\cdot a + t\cdot b$\footnote{선형 디오판토스 방정식 그중에서도 베주 항등식이라고도 한다.}
\end{center}
\end{justbox}

\subsection{베주의 항등식}\par
$ax + by =\gcd(x, y)$인 $a$, $b$가 존재한다.\par
일반해는 $\gcd(x ,y)$의 배수이다.
증명\par 
집합 $S = \left\{ m | m =ax+by> , x\in \Zeta , y \in \Zeta \right\}$를 생각해보면, 이 집합 $S$는 $S \subset \Zeta$ ,  $S \subset \varnothing$ ( x, y를 원소로 가짐을 알 수 있다.) 자연수의 정렬성으로부터 최소가 되는 원소 $d$가 존재한다.
$\alpha \in S \Rrightarrow \alpha = qd+r (0 \le r < d \because$ 나머지정리 )
만약 $d \nmid \algha$ 일때, $r > 0$,
$ r = \algha - qd , (\algha , d \in S)$ 
$\algha = a_{1}x+b_{1}y , d=a_{2}x+b_{2}y$라 하면. $r=(a_{1}-a{2}q)x + (b_{1}-qb{2})y \in S $
$0 < r < d$인 $r$에 대해 $d$가 최소라는 가정이 모순이다. $\therefore r = 0 , d \mid \algha (\forall \algha \in S)$
$ d \mid x, d \mid y \cdots$ $d$는  $x$ , $y$의 공약수
$\gcd(x, y)=k $라 할때, $d = akx^{''}+bky^{''}=k(ax^{''}+by^{''})$
$k \mid d$에서 $ k = d$
\subsection{활용}
이미 증명되어있는 유클리드 알고리즘의 흐름을 통해서 예시로 이해 해보자.\par
$a=273$  , $b=110$으로 하는 $\gcd(273,110)$을 구해봅시다.
\begin{center}
    $r= 273\bmod  110 = 53 \cdot\cdot\cdot\cdot \mathit{1}$
\end{center}
$a=110 , b=53$으로 지정
\begin{center}
    $r= 110\:\bmod \: 53 = 4\cdot\cdot\cdot\cdot \mathit{2}$
\end{center}
$a=53 , b=4$로 지정
\begin{center}
    $r= 53\:\bmod \: 4 = 1 \cdot\cdot\cdot\cdot \mathit{3}$
\end{center}
$a=4 , b=1$로 지정
\begin{center}
    $r= 4\bmod  1 = 0\cdot\cdot\cdot\cdot \mathit{4}$
\end{center}
$r=0$이므로 $\gcd(273,110)$은 최대공약수로 1을 가진다.
여기서 

$\mathit{4}$ 식으로 되돌아가면 이는 다음과 같이 쓸 수 있다.
\begin{center}
    $1=53 - 4\cdot13$
\end{center}
계속 역순으로 뒤집어 올라가자 $\mathit{3}$
\begin{center}
    $4=110 - 53\cdot2$
\end{center}
이를 처음의 식에 대입하면
\begin{center}
    $1=53 - (110 - 53\cdot2)\cdot13 =27\cdot53-13\cdot110 $
\end{center}
$\mathit{2}$
\begin{center}
    $53=273 - 110\cdot2$
\end{center}
이 식을 다시 대입하면
\begin{center}
    $1=27\cdot53-13\cdot110=27\cdot(273 - 110\cdot2)-13\cdot110=27\cdot273-67\cdot 110$
\end{center}
따라서 $s=27, t=-67$로서 성립하는 값을 찾았다.
\section{나머지 연산에서 곱셈에 대한 역원\footnote{역원: a와 연산자에 대해 연산결과가 항등원($=1$)이 되는 유일한 원소 b를 a의 역원이라한다.}(modular multiplicative inverse)}
$\gcd(n,\phi)=1$인 두 정수 $n>0, \phi>1$가 있다고 하자.\footnote{한 마디로 n과 $\phi$는 서로소이다.}
$n\cdot s\bmod  \phi =1 $을 만족시키는 $s$를 $n\bmod  \phi$의 역원(inverse) 이라고 한다.\newline
이 s를 효율적으로 구하는 방법에대해서 논해보자\par
$\gcd(n,\phi)=1$은 확장된 유클리드 알고리즘을 이용하여 $s'\cdot n + t \cdot \phi = 1$이되는 $s'$과 $t'$을 구할수있다. $n\cdot s'= -t'\phi+1$이 되고 $\phi>1$이므로 1이 나머지가 된다.
$n\cdot s'\bmod  \phi =1$에서 $s= s'\bmod  \phi$라 하면 $0 \le s <\phi$가 되며 $s  \ne 0$이다\par.위 식을 변형하면 .$s'=q\cdot \phi +s$ 가되며 이를 만족하는 정수 q가 존재한다 \par따라서 
\begin{center}
    $n\cdot s=ns'-\phi nq=-t'\phi +1 -\phi nq=\phi(-t'-nq)+1 $
\end{center}
따라서 $n\cdot s\bmod  \phi =1 $이 된다.
\newpage
\chapter{}
\section{오일러의 $\varphi$함수(Euler’s phi (totient) function)}
양의 정수 $n$에 대해서 $\varphi (n)$은 1부터 n까지의 양의 정수 중에 n과 서로소인 것의 개수를 나타내는 함수이다.\par\par

$\varphi (n)$은 다음의 성질이 있다.
\begin{itemize}
\item{소수 $p$에 대해서  $\varphi (p)=p-1$}
\item{ m, n이 서로소인 정수일 때,   $\varphi (mn)=\varphi (m)\varphi (n)$ }
\end{itemize}

성질이 몇 개 더 있지만 RSA에서 필요한것만을 다루기위해서 생략하였다.
첫번째 성질은 어찌보면 당연하다 $p$는 소수이니 자기 자신을 제외한 모든 수와 서로소이다 (여기서 1도 세야한다.)\par
두번째 성질은 두수의 곱 mn은 각각 m에대해서 나눠지는 수가 n개이고 n에 대해서 나눠지는 수가 m개 이며 mn으로 나눠지는 수가 한 개이므로 
$mn -\dfrac{mn}{m}-\dfrac{mn}{n}+\dfrac{mn}{mn} =mn -m -n +1=(m-1)(n-1)=\varphi (m)\varphi(n)$가 된다.

\section{오일러 정리(Euler's theorem)} \footnote{페르마의 소정리는 오일러 정리에서의 특수한 경우이다.}
\begin{justbox}
임의의 정수 a와 n이 서로소일 때, $a^{\varphi(n)} \bmod n = 1$
\end{justbox}
정수 n에 대해서 1부터 n까지의 양의 정수 중에 n과 서로소인 것의 집합을 생각해보자.
그러면 이는 집합
\begin{center}
    $A = \{ r_1 ,r_2,r_3,,,,r_{\varphi(n)}\}$\footnote{이러한 집합을 기약잉여계라고 부른다. 또한 집합 A의 원소의 갯수는 $\varphi(n)$이다.}
\end{center}
으로 나타낼 수 있다. 이 집합은 A라하고 이 각 원소에 n과 서로소인 a를 곱한 집합을 B집합이라 하자.
\begin{center}
    $B = \{ ar_1 ,ar_2,ar_3,,,,ar_{\varphi(n)}\} $
\end{center}
확실한건 $B$에 있는 모든 원소는 $n$과 서로소인 것이다. 그럼 $B$집합의 각 원소를 $\bmod n$에 대해 계산한 것을 생각해보자 이는 각 원소의 나머지가 a를 곱하기전 값과 같은지는 모르지만 $\varphi(n)$개에 대해서 각각 일대일대응이 가능 한다는것을 알수있다. \footnote{실제 증명은 귀류법을 통해서 증명할수있다.$ar_i  \equiv ar_j \bmod n $ 인 $1 \le i < j \le \varphi(n)$ 이 존재한다고 가정해보자.}
따라서 $A$의 모든 원소를 곱한 값에 $\bmod n$을 한것과 $B$의 모든 원소를 곱한 값에 $\bmod n$을 한 값은 같다.
\begin{center}
    $ar_1 \cdot ar_2 \cdot ar_3 \cdot\cdot\cdot ar_{\varphi}\bmod n = r_1 \cdot r_2 \cdot r_3 \cdot\cdot\cdot r_{\varphi} \bmod n$
\end{center}
\begin{center}
    $a^{\varphi(n)}\bmod n= 1$
\end{center}

\newpage
\chapter{}
\section{RSA \footnote{로널드 라이베스트(Ron Rivest), 아디 샤미르(Adi Shamir), 레너드 애들먼(Leonard Adleman)이 세명의 이름 앞글자를 따서 지었다.} 공개 키 암호 시스템(RSA public-key cryptosystem)}

이 알고리즘은 보안 기법중 하나로 가장 흔한 예시로서는 공인인증서가 있다.
\begin{center}
    $A \longrightarrow B$
\end{center}
$A$가 $B$에게 숫자를 하나 보낸다고 생각 해보자. $A$에게는 공개키가 필요하며 $B$에게는 개인키가 있어야한다. 공개키는 누가 가져도 상관없는 키이며 개인키는 절대로 노출되어서는 안되는 키이다.\par
$A$는 $B$에게 $a$를 보낼때 공개키를 이용하여 $a$를 $c$로 암호화 하여 보내며 $B$는 $c$를 공개키와 개인키를 이용하여 $a$로 복호화하여 읽는 방식이다.
\subsection{공개키, 암호키 생성}
두 개의 소수 $p,q$를 선택하여 $z=pq$를 계산한다.\footnote{그 후 $p ,q$는 버린다. 가지고 있어봤자 개인키가 뚫리는 취약점이 될수가있다.} 그 후 $\phi =(p-1)(q-1)$을 계산하고 $\gcd(n,\phi)=1$인 정수 n을 선택한다. 그후 $z$와 $n$을 공개한다. $ns\bmod \phi =1$이고 $0<s<\phi$를 만족시키는 s를 생성하여 s를 개인키로 사용한다.\footnote{s는 위에서 언급한 나머지 연산에서 곱셈에 대한 역원을 구하는 방법으로 효율적으로  구할수있다.}\par
\subsection{단계}
$A$가 $B$에게 정수 $a(0\le a\le z-1)$를 보내기 위해서 $A$는 $c=a^n \bmod z$ 를 계산하여 $c$를 보낸다.\footnote{c를 효율적으로 구하는 방법 또한 위에서 다루었다.}
$B$는 $c^s \bmod z$를 계산하면 이 값이 $a$이다.\par
\subsection{복호화 과정}

$ ns\bmod \phi =1 \Longleftrightarrow ns = b\varphi(n)+1$($b$는 어떤 상수)
\newline 
$c^s \bmod z=(a^n \bmod z)^s \bmod z = (a^n)^s \bmod z \newline = a^{ns}\bmod z =$
$a^{b\varphi(n)+1}\bmod z =(a^{\varphi(n)} \bmod z)^{b} a \bmod z =a$ \footnote{오일러정리 사용}

\subsection{이게 과연 안전한가?}


\newpage
참고 저서: 이산 수학(Discrete Mathematics) (저자:Richard Johnsonbaugh)\par
암호학과 네트워크 보안(cryptography and network security) (저자: Behrouz A. Forouzan)


\url{ https://blog.naver.com/mindo1103/221234421987}
\newline
\url{https://blog.naver.com/mindo1103/221234421626}
\end{document}\:
